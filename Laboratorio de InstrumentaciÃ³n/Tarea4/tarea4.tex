\documentclass[conference]{IEEEtran}
\IEEEoverridecommandlockouts
% The preceding line is only needed to identify funding in the first footnote. If that is unneeded, please comment it out.
\usepackage{amsmath,amsthm,amssymb} %modos matemáticos y  simbolos
\usepackage{latexsym,amsfonts} %simbolos matematicos
\usepackage{cancel} %hacer la linea que cancela las ecuaciones
\usepackage[spanish, es-noshorthands]{babel} %comandos en español y cambia el cuadro por la tabla
\decimalpoint %cambia las comas por puntos decimal
\usepackage[utf8]{inputenc} %caracteristicas del español
\usepackage{physics} %Simbolos fisicos
\usepackage{array} %mejores formatos de tabla
\parindent =0cm %sangria 
\usepackage{algorithmic}
\usepackage{graphicx}
\usepackage{textcomp}
\usepackage{xcolor}
\usepackage{mathtools} 
\usepackage[framemethod=TikZ]{mdframed}%Entornos talegas
\usepackage[colorlinks = true,
			linkcolor = blue,
			citecolor = black,
			urlcolor = blue]{hyperref}%formato de los links y URL's
\usepackage{multicol} %varias columnas
\usepackage{enumerate} %enumeraciones
\usepackage{pgf,tikz,pgfplots} %documentos en formato tikz
\usepackage{mathrsfs} %letras chingonas (transformada de laplace)
\usepackage{subfigure} %varias figuras seguidas
\usepackage{tabulary}
\usepackage{multirow} %ocupar varias filas en una tabla
\usepackage{fancybox} %recuadros talegas
\usepackage{float} %ubicar graficas
\usepackage{color}
\usepackage{comment}
\usepackage{stackrel}
\usepackage{calligra}
\usepackage{lipsum}
\usepackage{cite}
%\pgfplotsset{compat=1.17} 

\newcommand{\R}{\mathbb{R}}
\newcommand{\Z}{\mathbb{Z}}
%%%%%%%%%%%%%%%%%%%%%%%%%%%%%%%%%%%%%%%%%%%%%%%%%%%%%%
\def\BibTeX{{\rm B\kern-.05em{\sc i\kern-.025em b}\kern-.08em
    T\kern-.1667em\lower.7ex\hbox{E}\kern-.125emX}}
\begin{document}



%%% CARÁTULA
\begin{titlepage}



\begin{flushleft}
    Universidad de San Carlos de Guatemala \\
    Escuela de Ciencias Físicas y Matemáticas \\
    Curso: Laboratorio de Instrumentación \\
    Profesor: Wendy Miranda
\end{flushleft}

\vspace{6cm}

\begin{center}
    \huge{Amplificadores AO} \\[1cm]
    \large{Tarea 4}
\end{center}

\vspace{10.5cm}

\begin{flushright}
    Diego Sarceño \\
    $201900109$
\end{flushright}

\vspace{0.5cm}

\begin{center}
    Guatemala, 02 de noviembre del 2022
\end{center}

\end{titlepage}



\begin{abstract}
    
\end{abstract}

\begin{IEEEkeywords}
    
\end{IEEEkeywords}

\section{Objetivos}

\subsection{General}
    \begin{enumerate}[1.]
        \item 
    \end{enumerate}
\subsection{Específicos}
    \begin{enumerate}
        \item 
    \end{enumerate}
%\section{Introducción}
    
\section{Marco Teórico}
    
\section{Diseño Experimental}
    \subsection{Materiales a Utilizar}
        \begin{itemize}
    	\item 
    \end{itemize}

    \subsection{Procedimientos}
        \begin{enumerate}
            \item 
        \end{enumerate}
\section{Resultados}
    
\section{Discusión de Resultados}
\begin{enumerate}
    \item 
   
\end{enumerate}
\section{Conclusiones}
\begin{enumerate}
    \item 
\end{enumerate}
%\section{Recomendaciones}

\section{Anexos}

\begin{thebibliography}{00}
\bibitem{b1} Mano, M., 2003. \textit{Diseño Digital}. 3rd ed. México: PEARSON EDUCACIÓN.
\bibitem{b2} 2021. \textit{Circuit Diagram}. \url{https://www.circuit-diagram.org/}
\end{thebibliography}

\end{document}