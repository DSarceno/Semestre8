



\begin{ejercicio}
	Sabiendo que los operadores se pueden escribir en términos del los operadores escalera
		\begin{align*}
			L_x &= \frac{1}{2} \qty(L_+ + L_-) \\
			L_y &= \frac{1}{2} \qty(L_- - L_+),
		\end{align*}
	Entonces, es claro que $\expval{L_x} = 0$ dado que $\expval{L_+} = \expval{L_-} = 0$. Ahora $L_x ^2 = \frac{1}{4} \qty(L_+ ^2 + L_+ L_- + L_- L_+ + L_- ^2)$, por lo anterior $\expval{L_+ ^2} = \expval{L_- ^2} = 0$, entonces
		$$ \expval{L_x ^2} = \frac{1}{4} \expval{L_+ L_- + L_- L_+}, $$
		$$ \expval{L_y ^2} = -\frac{1}{4} \expval{-L_+ L_- - L_- L_+}, $$
	lo que implica que $\expval{L_x ^2} = \expval{L_y ^2}$. Calculando su valor, tomamos la relación $\expval{L_x ^2} = \expval{L_y ^2} = \frac{1}{2} \qty[\expval{L^2} - \expval{L_z ^2}]$, entonces
		$$ \expval{L_x ^2} = \expval{L_y ^2} = \frac{\hbar ^2}{2} \qty[l(l + 1) - m^2]. $$

\end{ejercicio}













\begin{ejercicio}
	Partiendo de la definición de los operadores
		\begin{align*}
			L_x &= i\hbar \qty(Z\pdv{y} - Y\pdv{z}) \\
			L_y &= i\hbar \qty(X\pdv{z} - Z\pdv{x}) \\
			L_z &= i\hbar \qty(Y\pdv{x} - X\pdv{y}).
		\end{align*}
	Utilizando la regla de la cadena y trasladando a coordenadas esfericas
		{\begin{align*}
			\begin{split}
				\pdv{r}{x} &= \sen{\theta} \cos{\phi} \\
				\pdv{r}{y} &= \sen{\theta} \sin{\phi} \\
				\pdv{r}{z} &= \cos{\theta} \\
			\end{split}
			&&
			\begin{split}
				\pdv{\theta}{x} &= \frac{1}{r} \cos{\theta} \cos{\phi} \\
				\pdv{\theta}{y} &= \frac{1}{r} \cos{\theta} \sin{\phi} \\
				\pdv{\theta}{z} &= -\frac{1}{r} \sin{\theta} \\
			\end{split}
			&&
			\begin{split}
				\pdv{\phi}{x} &= -\frac{1}{r} \frac{\sin{\phi}}{\sin{\theta}} \\
				\pdv{\phi}{y} &= \frac{1}{r} \frac{\cos{\phi}}{\sin{\theta}} \\
				\pdv{\phi}{z} &= 0. \\
			\end{split}
		\end{align*}}
		
	reemplazando en la definición de los operadores junto con la regla de la cadena. Por lo tanto, se tiene
	\begin{align*}
		L_x = i\hbar \qty(\sin{\phi} \pdv{\theta} + \cot{\theta} \cos{\phi} \pdv{\phi}),
	\end{align*}
	lo mismo para $y$
	\begin{align*}
		L_y = i\hbar \qty(-\cos{\phi} \pdv{\theta} + \cot{\theta} \sin{\phi} \pdv{\phi})
	\end{align*}
	y
	\begin{align*}
		L_z = -i\hbar \pdv{\phi}.
	\end{align*}
	Ahora, encontrando el cuadrado de cada componente, se tiene
		\begin{align*}
			L_x ^2 &= -\hbar ^2 \qty(\sin{\phi} \pdv{\theta} + \cot{\theta} \cos{\phi} \pdv{\phi}) \qty(\sin{\phi} \pdv{\theta} + \cot{\theta} \cos{\phi} \pdv{\phi}), \\
			L_y ^2 &= -\hbar ^2 \qty(-\cos{\phi} \pdv{\theta} + \cot{\theta} \sin{\phi} \pdv{\phi}) \qty(-\cos{\phi} \pdv{\theta} + \cot{\theta} \sin{\phi} \pdv{\phi}), \\
			L_z ^2 &= -\hbar ^2 \pdv[2]{\phi}.
		\end{align*}
	Luego de un poco de álgebra, se llega a 
		$$ L^2 = -\hbar ^2 \qty[\frac{1}{\sin{\theta}} \pdv{\theta} \qty(\sin{\theta} \pdv{\theta}) + \frac{1}{\sin ^2 {\theta}} \pdv[2]{\phi}]. $$
\end{ejercicio}














\begin{ejercicio}
	Solución:
	\begin{enumerate}[a)]
		\item 
		\item Sabemos que $\braket{r,\theta ,\phi}{l,m}$, lo podemos escribir en términos de los armónicos esféricos $Y^m _l$. Estos los encontramos resolviendo la ecuación $L^2 \psi = \lambda \psi$ por separación de variables. Las soluciones resultan $\Theta (\theta) = P^m _l (\cos{\theta})$ y $\Phi (\phi) = e^{im\phi}$. Por propiedades de los polinomios de Legendre, la constante de normalización  es $C_{lm} = (-1)^m \sqrt{\qty(\frac{2l+1}{2}) \frac{(l-m)!}{(l+m)!}}$ y la de $\Psi$ es $\frac{1}{2\pi}$. Entonces, los armónicos esféricos se pueden escribir como
			$$ \braket{r,\theta ,\phi}{l,m} = Y_l ^m (\theta ,\phi) = (-1)^m \sqrt{\frac{(2l + 1) (l-m)!}{4\pi (l + m)!}} P^m _l (\cos{\theta}) e^{im\phi}. $$
		\item Con el inciso anterior calculamos $\braket{r,\theta ,\phi}{2,1}$ y $\braket{r,\theta ,\phi}{2,2}$ únicamente valuando $l,m$ en $Y^m _l$. Entonces (utilizando la tabla $5.1$ y $5.2$ de Zettili.)
		\begin{align*}
			\braket{r,\theta ,\phi}{2,1} &= Y_2 ^1 (\theta ,\phi) = -\sqrt{\frac{15}{8\pi}} e^{i\phi} \sin{\theta} \cos{\theta} , \\
			\braket{r,\theta ,\phi}{2,2} &= Y_2 ^2 (\theta ,\phi) = \sqrt{\frac{15}{32\pi}} e^{2i\phi} \sin ^2 {\theta} .
		\end{align*}
	\end{enumerate}
\end{ejercicio}














\begin{ejercicio}
	
\end{ejercicio}



















\begin{ejercicio}
	Sabiendo que
		$$ \ket{1,1} = \mqty(1 \\ 0 \\ 0), \qquad \qquad \ket{1,0} = \mqty(0 \\ 1 \\ 0), \qquad \qquad \ket{1,-1} = \mqty(0 \\ 0 \\ 1). $$
	Utilizando la función \texttt{Norm[]} y \texttt{Solve[]} se encuentra la constante de normalización la cuál es $A = \pm \frac{2}{\sqrt{7}}$. Con esto y por lo encontrado la tarea pasada, $\expval{L_x} = \expval{L_y} = 0$. Ahora, para los otros dos operadores
	\begin{align*}
		\expval{L_z}{\psi} &= -\frac{\hbar}{7} + \hbar \frac{2}{7} = \frac{\hbar}{7}, \\
		\expval{L^2}{\psi} &= \frac{2\hbar ^2}{7} + \frac{4}{7} \qty(2\hbar ^2) + \frac{2}{7} (2\hbar ^2) = 2\hbar ^2 .
	\end{align*}
\end{ejercicio}

























\begin{ejercicio}
	Dadas las relaciones
	$$ [J_x,J_y] = i\hbar J_z, \qquad \qquad [J_y,J_z] = i\hbar J_x, \qquad \qquad [J_z,J_x] = i\hbar J_y. $$
	Utilizando la propiedad $[AB,C] = A[B,C] + [A,C]B$, se pueden expandir los términos del lado izquierdo de la igualdad
	\begin{align*}
		[J_x J_y,J_z] &= i\hbar J_x ^2 - i\hbar J_y ^2, \\
		[J_x,J_y J_z] &= -i\hbar J_y ^2 + i\hbar J_z ,
	\end{align*}
	sumando y factorizando
	$$ = i\hbar \qty(J_x ^2 - 2J_y ^2 + J_z ^2). $$
	Concluyendo la demostración.
\end{ejercicio}











%%%%%