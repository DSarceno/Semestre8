\begin{ejercicio}
	Solución, expandiendo los operadores de momento angular y se utiliza la siguiente propiedad: $[AB,C] = A[B,C] + [A,C]B$.
	\begin{itemize}
		\item Demostrar $[L_x ,Y] = i\hbar Z$
			\begin{align*}
				&= [YP_z ,Y] - [ZP_y ,Y] \\
				&= Y[P_z ,Y] + [Y,Y]P_z - Z[P_z ,Y] - [Z,Y]P_y \\
				&= -Z[P_y ,Y] = -Z(-i\hbar) \\
				&= i\hbar Z.
			\end{align*}
		\item Demostrar $[L_x ,P_y] = i\hbar P_z$
			\begin{align*}
				&= [YP_z ,P_y] - [ZP_y ,P_y] \\
				&= Y[P_z ,P_y] + [Y,P_y]P_z \\
				&= i\hbar P_z.
			\end{align*}
		\item Demostrar $[L_x ,P^2] = 0$
			\begin{align*}
				&= [L_x ,P_x ^2 + P_y ^2 + P_z ^2] \\
				&= \underbrace{[YP_z,P_x ^2]}_{0} + \underbrace{[YP_z,P_y ^2]}_{2i\hbar P_y P_z} + \underbrace{[YP_z,P_z ^2]}_{0} - \underbrace{[ZP_y,P_x ^2]}_{0} - \underbrace{[ZP_y,P_y ^2]}_{0} - \underbrace{[ZP_y,P_z ^2]}_{2i\hbar P_z P_y} \\
				&= 2i\hbar [P_y,P_z] = 0
			\end{align*}
	\end{itemize}
\end{ejercicio}




\begin{ejercicio}
	
\end{ejercicio}




\begin{ejercicio}
	
\end{ejercicio}




\begin{ejercicio}
	Sabiendo que
		$$ L^2 \ket{l,m} = \hbar ^2 l(l+1)\ket{k,m} $$
		$$ L_z \ket{l,m} = \hbar m\ket{l,m}. $$
	Ya se tiene la base propia, teniendo que $m\in [-l,l]$ entonces GRAFICA!
%	\begin{figure}[H]
%		\centering
%		\includegraphics[scale=0.5]{img/proper_basis.png}
%		\caption{Base Propia de $L^2$.}
%		\label{properbasis}
%	\end{figure}
	Además, se tiene que 
		\begin{align*}
			L_+ \ket{l,m} &= C_+ \ket{l,m} \\
			L_- \ket{l,m} &= C_- \ket{l,m},
		\end{align*}
	cuyas constantes (según Zettili) son
		\begin{align*}
			C_\pm &= \hbar \sqrt{l(l + 1) - m(m \pm 1)}.
		\end{align*}
	Encontrando las representaciones matriciales ($m' \neq m$ y $l = l' = 1$) $\mel{l,m}{\hat{A}}{l',m'}$
		\begin{align*}
			\mel{1,0}{L_-}{1,1} &= \sqrt{2} \hbar \\
			\mel{1,-1}{L_-}{1,0} &= \sqrt{2} \hbar \\
			L_- &= \frac{\hbar}{\sqrt{2}} \mqty(0 & 0 & 0 \\ 1 & 0 & 0 \\ 0 & 1 & 0)
		\end{align*}
	el resto son cero.
		\begin{align*}
			\mel{1,1}{L_+}{1,0} &= \sqrt{2} \hbar \\
			\mel{1,0}{L_+}{1,-1} &= \sqrt{2} \hbar \\
			L_+ &= \frac{\hbar}{\sqrt{2}} \mqty(0 & 1 & 0 \\ 0 & 0 & 1 \\ 0 & 0 & 0),
		\end{align*}
	$L_z$ es más claro por su definición
\begin{align*}
			\mel{1,1}{L_z}{1,1} &= \hbar \\
			\mel{1,0}{L_z}{1,0} &= 0 \\
			\mel{1,-1}{L_z}{1,-1} &= -\hbar \\
			L_z &= \hbar \mqty(1 & 0 & 0 \\ 0 & 0 & 0 \\ 0 & 0 & 1).
		\end{align*}
	Teniendo la relación entre los operadores escalera y $L_y$ y $L_x$
		$$ L_x = \frac{1}{2} \qty(L_+ + L_-), $$
		$$ L_y = \frac{i}{2} \qty(L_- - L_+). $$
	Reemplazando y sumando las matrices, se tiene
		$$ L_x = \frac{\hbar}{2\sqrt{2}} \mqty(0 & 1 & 0 \\ 1 & 0 & 1 \\ 0 & 1 & 0), $$
		$$ L_y = \frac{\hbar}{2\sqrt{2}} \mqty(0 & -1 & 0 \\ 1 & 0 & -1 \\ 0 & 1 & 0). $$
	Y $L^2 = L_x ^2 + L_y ^2 + L_z ^2$, utilizando mathematica
		$$ L^2 = \frac{\hbar ^2}{4} \mqty(5 & 0 & 0 \\ 0 & 2 & 0 \\ 0 & 0 & 4). $$
\end{ejercicio}






\begin{ejercicio}
	
\end{ejercicio}














%%%%%