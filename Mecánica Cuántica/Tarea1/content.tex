\begin{ejercicio}
	 Teniendo las definiciones de los operadores de creación y aniquilación 
	$$
		\left.\begin{array}{ccc}
			a\ket{n} & = & \sqrt{n} \ket{n - 1} \\
			a^\dagger \ket{n} & = & \sqrt{n + 1} \ket{n + 1} \\
		\end{array}\right\}
		\Rightarrow \left\{\begin{array}{ccc}
			a^2 \ket{n} & = & \sqrt{n(n - 1)} \ket{n - 2} \\
			a^{\dagger ^{2}} \ket{n} & = & \sqrt{(n + 1)(n + 2)} \ket{n + 2} 
		\end{array}\right. .
	$$
	Calulamos los operadores $X$ y $P$ en términos de $a$ y $a^\dagger$
	\begin{align*}
		X &= \sqrt{\frac{\hbar}{2m\omega}} \qty(a^\dagger + a) \\
		P &= m\omega i \sqrt{\frac{\hbar}{2m\omega}} \qty(a^\dagger - a) \\
		X^2 &= \frac{\hbar}{2m\omega} \qty(a^{\dagger ^{2}} + a^\dagger a + aa^\dagger + a^2) \\
		P^2 &= -\qty(\frac{m\omega \hbar}{2})^2 \qty(a^{\dagger ^{2}} + a^2 - a^\dagger a - aa^\dagger) .
	\end{align*}
	Encontrando
	\begin{align*}
		\mel{m}{X}{n} &= \mel{m}{\sqrt{\frac{\hbar}{2m\omega}} \qty(a^\dagger + a)}{n} \\
			&= \sqrt{\frac{\hbar}{2m\omega}} \qty(\sqrt{n + 1} \braket{m}{n + 1} + \sqrt{n} \braket{m}{n - 1}) \\
			\Aboxed{ &= \sqrt{\frac{\hbar}{2m\omega}} \qty(\sqrt{n + 1} \delta_{m,n + 1} + \sqrt{n} \delta_{m,n - 1})}
	\end{align*}
	bajo la misma idea
	\begin{align*}
		\Aboxed{ \mel{m}{P}{n} &= \sqrt{\frac{\hbar m \omega}{2}} i \qty( \sqrt{n + 1} \delta_{m,n + 1} - \sqrt{n} \delta_{m,n - 1} ) }.
	\end{align*}
	\begin{align*}
		\mel{m}{X^2}{n} &= \frac{\hbar}{2m\omega} \bra{m} \qty[\sqrt{(n + 1)(n + 2)} \ket{n + 2} + \sqrt{n(n - 1)} \ket{n - 2} + n\ket{n} + (n + 1)\ket{n}] \\
		\Aboxed{ &= \frac{\hbar}{2m\omega} \qty(\sqrt{(n + 1)(n + 2)} \delta_{m,n + 2} + \sqrt{n(n - 1)} \delta_{m,n - 2} + n\delta_{m,n} + (n + 1)\delta_{m,n}) }
	\end{align*}
	\begin{align*}
		\Aboxed{ \mel{m}{P^2}{n} &= -\frac{m\omega \hbar}{2} \qty( \sqrt{(n + 1)(n + 2)}\delta_{m,n + 2} + \sqrt{n(n - 1)} \delta_{m,n - 2} - n\delta_{m,n} - (n + 1) \delta_{m,n} ) }
	\end{align*}
	Encontrando $XP$ y $PX$, se tiene
	$$ XP = \frac{\hbar}{2} i \qty( \sqrt{(n + 1)(n + 2)} \delta _{m,n + 2} - n\delta _{m,n} + (n + 1) \delta _{m,n + 1} - \sqrt{n(n - 1)} \delta _{m,n - 2} ), $$
	$$ PX = \frac{\hbar}{2} i \qty( \sqrt{(n + 1)(n + 2)} \delta _{m,n + 2} + n\delta _{m,n} - (n + 1) \delta _{m,n + 1} - \sqrt{n(n - 1)} \delta _{m,n - 2} ). $$
	entonces
	$$ \boxed{\mel{m}{XP + PX}{n} = \hbar i \qty(\sqrt{(n + 1)(n + 2)} \delta _{m,n + 2} - \sqrt{n(n - 1)} \delta _{m,n - 2}).} $$
\end{ejercicio}




\begin{ejercicio}
	 Teniendo el estado inicial
	 	$$ \ket{\psi _o} = \sum _{n = 0} ^\infty c_n \ket{n}, $$
	 para encontrar en todo instante $t$
	 	$$ \ket{\psi _(t)} = U(t) \ket{\psi _o}, \qquad \qquad U(t) = e^{-\frac{it}{\hbar} H}. $$
	 Sabiendo que $e^{-\frac{it}{\hbar} H} \ket{n} = e^{-i\omega (n + 1/2) t} \ket{n} $, entonces
	 	$$ \ket{\psi (t)} = \sum _{n = 0} ^\infty c_n e^{-i\omega (n + 1/2) t} \ket{n} = \sum _{n = 0} ^\infty c_n e^{-i\omega n t} \ket{n} $$
	 podemos eliminar el término $e^{-i\frac{\omega}{2} t}$ dado que no afecta a la representación del estado.
	 	$$ \Psi (x,t) = \braket{x}{\psi (t)} = \sum _{n = 0} ^\infty c_n e^{-i\omega n t} \underbrace{\braket{x}{n}}_{\phi _n (x)}. $$
\end{ejercicio}




\begin{ejercicio}
	 Teniendo el operador $X = \sqrt{\frac{\hbar}{2m\omega}} (a^\dagger + a)$. Multiplicando por si mismo
	 	$$ X^2 = \frac{\hbar}{2m\omega} \underbrace{(a^\dagger + a)(a^\dagger + a)}_{a^{\dagger ^2} + a^2 + a^\dagger a + aa^\dagger}, $$
	 	$$ X^2 = \frac{\hbar}{2m\omega} \qty(a^{\dagger ^2} + a^2 + a^\dagger a + aa^\dagger). $$
\end{ejercicio}




\begin{ejercicio}
	 Teniendo los operadores
	 	$$
		\begin{array}{c}
			X^2 = \frac{\hbar}{2m\omega} \qty(a^{\dagger ^2} + a^2 + a^\dagger a + aa^\dagger) \\
			P^2 = -\frac{m\hbar \omega}{2} \qty(a^{\dagger ^2} + a^2 - a^\dagger a - aa^\dagger) \\
			H = \hbar \omega \qty(a^\dagger a + \frac{1}{2}).
		\end{array}			 	
	 	$$
	 y teniendo que $\expval{a^2}{n} = \expval{a^{\dagger ^2}}{n} = 0$, $\expval{a^\dagger a}{n} = n$ y $\expval{aa^\dagger}{n} = n + 1$. Entonces
	 	$$ \expval{X^2} = \frac{\hbar}{2m\omega} (2n + 1), $$
	 	$$ \expval{P^2} = \frac{m\hbar \omega}{2} (2n + 1), $$
	 	$$ \expval{H} = \hbar \omega \qty(n + \frac{1}{2}). $$
	 Valuando para el estado base ($n = 0$) se tiene
	 	$$ \expval{X^2} = \frac{\hbar}{2m\omega}, $$
	 	$$ \expval{P^2} = \frac{m\hbar \omega}{2}, $$
	 	$$ \expval{H} = \frac{\hbar \omega}{2}. $$
	 $\hfill \square$
\end{ejercicio}




\begin{ejercicio}
	 Teniendo la energía para el oscilador cuántico $E_n = \hbar \omega \qty(n + \frac{1}{2})$, tomando la energía del oscilador armónico $E = \frac{1}{2} kA^2$. Entonces, en términos de $\omega$
	 	$$ A_n = \sqrt{\frac{\hbar (2n + 1)}{m \omega}}. $$
	 El parámetro $\lambda$ es
	 	$$ \lambda = \sqrt{\frac{\hbar}{m\omega}}. $$
	 La probabilidad fuera del límite clásico es
	 	$$ P_{out} = 1 - P_{in} \qquad \Rightarrow \qquad P_{out} = 1 - \int _{-A_0} ^{A_0} \abs{\psi _0 (x)} \dd{x}, $$
	 para el estado base se tiene
	 	$$ \psi _0 (x) = \qty(\frac{m\omega}{\pi \hbar})^{\flatfrac{1}{4}} e^{-\frac{m\omega x^2}{2\hbar}}; $$
	 por lo tanto
	 	$$ P_{out} = 1 - \qty(\frac{m\omega}{\pi \hbar})^{\flatfrac{1}{2}} \int _{-A_0} ^{A_0} e^{-\frac{m\omega x^2}{\hbar}} \dd{x} \qquad \Rightarrow \qquad P_{out} = 1 -\erf{1} = 15.73\%. $$
\end{ejercicio}




\begin{ejercicio}
	 Dado lo encontrado el ejercicio anterior, encontramos la amplitud para el estado $\ket{1}$ y la probabilidad fuera del límite clásico.
	 	$$ A_1 = \sqrt{\frac{3\hbar}{m\omega}}, $$
	 la función de onda es
	 	$$ \psi _1 (x) = \frac{\sqrt{2}}{\pi ^{\frac{1}{4}}} \qty(\frac{m\omega}{\hbar})^{\flatfrac{3}{4}} x e^{-\frac{m\omega x^2}{2\hbar}}. $$
	 Integrando
	 	$$ P_{out} = 1 - \frac{2}{\sqrt{\pi}} \qty(\frac{m\omega}{\hbar})^{\flatfrac{3}{2}} \int _{-A_1} ^{A_1} x^2 e^{-\frac{m\omega x^2}{\hbar}} \dd{x} = P_{out} = 1 - \qty(\text{erf}\left(\sqrt{3}\right)-2 \sqrt{\frac{3}{\pi }} e^{-3}) = 11.16\%. $$
\end{ejercicio}










%%%%%%%